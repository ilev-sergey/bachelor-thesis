%% Согласно ГОСТ Р 7.0.11-2011:
%% 5.3.3 В заключении диссертации излагают итоги выполненного исследования, рекомендации, перспективы дальнейшей разработки темы.
%% 9.2.3 В заключении автореферата диссертации излагают итоги данного исследования, рекомендации и перспективы дальнейшей разработки темы.
В работе было проведено исследование возможных физических механизмов, влияющих на сегнетоэлектрические свойства конденсаторов на основе оксида гафния-циркония. На примере исследования микроскопических и функциональных свойств структур W/Hf\(_{0.5}\)Zr\(_{0.5}\)O\(_2\)/Pt методами АСМ, РЭМ, XRD, PFM и электрофизическими измерениями было показано, что параметры, используемые при импульсном лазерном напылении, могут существенно влиять на функциональные свойства ячеек сегнетоэлектрической памяти.

В ходе работы были получены следующие экспериментальные результаты: установлено, что в структурах, изготовленных при напылении платинового электрода лазерными импульсами большей энергии, наблюдается увеличение размера доменов, уменьшение величины встроенного электрического поля и увеличение остаточной поляризации. Физический механизм, объясняющий экспериментальные результаты, связан с влиянием размера зёрен платины, который увеличивается при увеличении энергии импульсов. Увеличение размера зёрен приводит к уменьшению длины границ зёрен, которые являются резервуаром для сегрегации атомов кислорода оксида гафния-циркония и через которые также возможна миграция атомов кислорода в окружающую среду. В результате, увеличение размера зёрен платины приводит к уменьшению плотности заряженных вакансий и величины диполя на верхней границе раздела, что в свою очередь уменьшает встроенное электрическое поле в пленке оксида гафния-циркония. Как известно, встроенное поле может значительно изменять сегнетоэлектрические свойства структур, в частности приводя к различной выраженности эффекта wake-up и различным значениям величины остаточной поляризации.

Кроме того, в работе был предложен метод для количественного сравнения вертикальной и латеральной компоненты пьезоотклика при измерениях с помощью микроскопии пьезоотклика. Метод был использован для восстановления трёхмерного вектора поляризации в функциональной структуре на основе оксида гафния. Было показано, что при переключении поляризации в большей части структуры происходит поворот вектора поляризации на \ang{180}, что объясняется наличием одной полярной оси в структурной орторомбической фазе \(Pca2_1\). Однако также наблюдаются области ферроупругого переключения, в которых поляризация переключается не на \ang{180}, что объясняется локальными механическими напряжениями, сохраняющимися даже после проведения процедуры электротренировки. Кроме того, установлено, что латеральный пьезоэлектрический отклик превышает вертикальный, что говорит о перспективности применения данного материала для разработки устройств на основе гетероструктур ферромагнетик-сегнетооэлектрик, принцип действия которых основан на обратном магнитострикционнном эффекте. Таким образом, предложенный метод может быть полезным как в исследовании природы сегнетоэлектрических свойств в новых материалах, так и в более прикладных задачах.

