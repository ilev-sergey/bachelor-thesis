\chapter{Сегнетоэлектрические материалы}\label{ch:ch2}

\section{Моделирование процесса напыления}\label{sec:ch2/sec2}
% Для бла-бла было произведено моделирование 
Для моделирования взаимодействия налетающих частиц платины с HZO были произведены квантовомеханические расчёты столкновений на основе метода Монте-Карло с помощью пакета Stopping and Range of Ions in Matter (SRIM) \cite{zieglerSRIMStoppingRange2010}. Оценка энергии налетаемых частиц была получена с помощью расчёта кинетической энергии атомов при лазерной абляции \cite{mutaevVLIYaNIEVERHNEYGRANICY2020}:
\todo{добавить интуиции к формулке}
\[\bar{E} = 9.92 \cdot 10^{4} A^{1/8} \tau^{1/4} Z^{3/4} (Z+1)^{3/8} (I\lambda)^{1/2}k_b,\] где \(A\) -- атомная масса распыляемого элемента, \(\tau\) -- длительность импульса, \(Z\) -- средний заряд ионов, вылетающих при абляции, \(I\) -- интенсивность лазерного излучения в \si{\watt}/\si{\cm}\(^2\), \(\lambda\) -- длина волны лазера, \(k_b\) -- постоянная Больцмана.

При импульсном лазерном напылении использовался лазер Nd:YAG с длиной волны \(\lambda=\) \SI{1064}{\nano\meter}, длительностью импульсов \(\tau=\) \SI{10}{\nano\second}. Интенсивность оценивалась как \(I=\alpha(1-\beta)\frac{J}{S}\), где \(\alpha=(1-R)^{2N}\) -- коэффициент прохождения излучения через \(N=5\) элементов оптической системы, изготовленных из кварцевого стекла с отражательной способностью \(R=0.0337\) \cite{polyanskiyRefractiveindexInfoDatabase2024} для длины волны \(\lambda\) (поглощение при этом не учитывалось), \(\beta=0.748\) -- отражательная способность платины \cite{weberHandbookOpticalMaterials2003}, \(J\) -- энергия одного импульса, \(S\) -- площадь лазерного пятна. Для оценки также использовался характерный средний заряд ионов Pt при лазерном напылении $Z=0.56$ \cite[с.~141]{easonPulsedLaserDeposition2007}

\begin{table} [htbp]
    \centering
    \begin{threeparttable}% выравнивание подписи по границам таблицы
        \caption{Название таблицы}\label{tab:Ts0Sib}%
        \begin{tabular}{ | p{2.5cm} | p{3cm} | p{3cm} | p{2.3cm}l | }
            \hline
            \hline
            \centering \(J\), \si{\milli\joule} & \centering \(I\), \si{\giga\watt}/\si{\cm}\(^2\) & \centering \(\bar{E}\), \si{\electronvolt} & \centering \(d\), \si{\nm} & \\
            \hline
            \centering 225                      & \centering  13.4                                 & \centering  151                            & \centering  1.3            & \\
            \centering 145                      & \centering  8.6                                  & \centering  121                            & \centering 1.2             & \\
            \centering 93                       & \centering  5.5                                  & \centering  97                             & \centering 1.1             & \\
            \centering 45                       & \centering  2.7                                  & \centering  68                             & \centering 1.0             & \\
            \hline
            \hline
        \end{tabular}
    \end{threeparttable}
\end{table}

На рисунке \cref{fig:pt_distr} показано \todo{вероятностное} распределение атомов платины в слое HZO при \todo{... многократном} ... Стоит отметить, что при распылении проникновение частиц платины в HZO быстро заканчивается: расстояние между плоскостями (111) в кубической гранецентрированной решётке составляет \(\frac{a}{\sqrt{3}}\approx\) \SI{2.26}{\angstrom}, поэтому структурные свойства HZO могут измениться лишь при напылении \(\sim 4-5\) первых атомных слоёв.

\begin{figure}[ht]
    \centerfloat{
        \hfill
        \subcaptionbox[List-of-Figures entry]{\label{fig:pt_distr_151eV} \SI{151}{\electronvolt}}{%
            \includegraphics[width=0.5\linewidth]{151 eV.png}}
        \hfill
        \subcaptionbox{\label{fig:pt_distr_121eV}\SI{121}{\electronvolt}}{%
            \includegraphics[width=0.5\linewidth]{121 eV.png}}
        \hfill
        \\
        \subcaptionbox{\label{fig:pt_distr_97eV}\SI{97}{\electronvolt}}{%
            \includegraphics[width=0.5\linewidth]{97 eV.png}}
        \hfill
        \subcaptionbox{\label{fig:pt_distr_68eV}\SI{68}{\electronvolt}}{%
            \includegraphics[width=0.5\linewidth]{68 eV.png}}
        \hfill
    }
    \caption[Этот текст попадает в названия рисунков в списке рисунков]{Распределение атомов платины в плёнке HZO \todo{после 100 000 актов распыления} при различных энергиях распыляемых частиц}\label{fig:pt_distr}
\end{figure}

\nomenclature{FEM}{finite element method, метод конечных элементов}
\nomenclature{HZO}{Hf0.5Zr0.5O2, оксид гафния-циркония}
\nomenclature{АСО}{атомно-слоевое осаждение}
\nomenclature{АСМ}{атомно-силовая микроскопия}
\nomenclature{PUND}{positive up negative down, электрофизический метод измерения остаточной поляризации}
\nomenclature{CMOS}{complementary metal-oxide-semiconductor, комплементарная структура металл-оксид-полупроводник}
\nomenclature{БТО}{быстрая термическая обработка}
\nomenclature{NVM}{non-volatile memory, энергонезависимая память}
\nomenclature{FIT}{finite integration technique, метод конечных интегралов}
\nomenclature{FMM}{fast multipole method, быстрый метод многополюсника}
\nomenclature{FVTD}{finite volume time-domain, метод конечных объёмов
    во~временной области}
\nomenclature{MLFMA}{multilevel fast multipole algorithm, многоуровневый
    быстрый алгоритм многополюсника}
\nomenclature{BEM}{boundary element method, метод граничных элементов}
\nomenclature{CST MWS}{Computer Simulation Technology Microwave Studio
    программа для компьютерного моделирования уравнен Максвелла}
\nomenclature{DDA}{discrete dipole approximation, приближение дискретиных
    диполей}

\FloatBarrier
