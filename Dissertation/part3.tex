\chapter{Методы исследования \todo{...}}\label{ch:ch3}

\section{Атомно-силовая микроскопия}\label{sec:ch3/sect1}
Атомно-силовая микроскопия (АСМ) позволяет осуществлять картирование рельефа поверхности образца, а также отклика различной природы с высоким пространственным разрешением. Сканирование осуществляется с помощью иглы с радиусом закругления \(\sim\)\SI{10}{\nm}, расположенной на конце кантилевера (рис. \cref{fig:afm_cantilever}). Отклонение кантилевера при взаимодействии иглы с образцом вызывает смещение лазерного луча, отражающегося от поверхности кантилевера, которое, в свою очередь, детектируется при помощи четырёхсекционного фотоприёмника (рис. \cref{fig:afm_system}).

\begin{figure}[ht]
    \centerfloat{
        \hfill
        \subcaptionbox[List-of-Figures entry]{\label{fig:afm_cantilever} \SI{151}{\electronvolt}}{%
            \includegraphics[width=0.5\linewidth]{afm_canttilever.png}}
        \hfill
        \subcaptionbox{\label{fig:afm_system}\SI{121}{\electronvolt}}{%
            \includegraphics[width=0.5\linewidth]{afm.png}}
        \hfill
    }
    \caption[Этот текст попадает в названия рисунков в списке рисунков]{Распределение атомов платины в плёнке HZO \todo{после 100 000 актов распыления} при различных энергиях распыляемых частиц}\label{fig:afm}
\end{figure}

\subsection{Микроскопия пьезоотклика}\label{sec:ch3/sect1/sub1}
Микроскопия пьезоотклика (piezoresponse force microscopy, PFM) позволяет исследовать распределение пьезоотклика в пьезоэлектрике с помощью измерения величины деформации, вызванной обратным пьезоэффектом (рис. \cref{fig:pfm}).

\begin{figure}[ht]
    \centerfloat{
        \includegraphics[width=0.6\linewidth]{pfm.jpg}
    }
    \caption{Микроскопия пьезоотклика}\label{fig:pfm}
\end{figure}

При этом возможно использование двух принципиально разных способов измерения: исследование пьезоотклика сегнетоэлектрического слоя при приложении напряжения между иглой и нижним электродом (\textit{ex citu}) и исследование пьезоотклика в готовой ячейке памяти (\textit{in situ}) при приложении напряжения между верхним и нижним электродами. При этом в случае \textit{ex citu} измерений сканируется непосредственно сегнетоэлектрическая плёнка, а в случае \textit{in situ} измерений сканируется поверхность электрода, деформирующаяся вслед за сегнетоэлектрическим слоем. Каждый из подходов имеет свои недостатки и преимущества. Так, в случае \textit{ex situ} измерений возможно \todo{протекание электрохимических реакций на поверхности сегнетоэлектрика [ref], неравномерное электрическое поле, которое к тому же в значительной степени зависит от формы иглы в момент измерения}, однако \textit{in situ} измерения неизбежно ухудшают пространственное разрешение ...,

\todo{Из-за слабого отклика появляется необходимость в использовании резонансных методик, а поскольку контактная резонансная частота изменяется при сканировании поверхности, возникает необходимость отслеживания резонансной частоты при измерениях, DART, BE PFM сравнение}

\subsubsection{Измерение латерального пьезоотклика}
В случае PFM измерений интересным становится использование не только сигнала, соответствующего вертикальному смещению пятна, но ... кручению кантилевера вдоль своей оси

\subsubsection{Спектроскопия}
(Switching spectroscopy, SS PFM)

\subsubsection{Количественные измерения}
Для получения количественных результатов при PFM измерениях необходимо провести калибровку \todo{прибора, кантилевер}. Для этого используется \todo{возбуждение кантилевера в воздухе} ... термопик