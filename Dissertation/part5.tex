\chapter{Физические механизмы, связывающие микроскопические характеристики с сегнетоэлектрическими свойствами структур на основе оксида гафния-циркония}

\todo{подумать насчёт структуры, лучше механизм/встроенное поле или встроенное поле/механизмы?}

\section{Влияние размера зёрен верхнего электрода}

Увеличение размера зёрен вследствие увеличения энергии импульса, используемого при импульсном лазерном напылении, предположительно приводит к увеличению скорости диффузии кислорода. Больший размер зёрен приводит к уменьшению длины границ зёрен, а значит к увеличению количества атомов кислорода, на месте которых может произойти образование кислородных вакансий. Повышенная концентрация заряженных дефектов, в свою очередь, приводит к образованию больших встроенных полей и пиннингу доменов.

\subsection{Встроенные поля}

Различная концентрация кислородных вакансий в плёнке HZO до циклирования приводит к образованию встроенных (built-in) электрических полей \(E_\text{bi}\) и пиннингу доменов \todo{ref}. При приложении внешнего электрического поля \(E_\text{ext}\), эффективное поле \(E_\text{eff}\) в СЭ слое складывается из суммы полей: \(\boldsymbol{E_\text{eff}} = \boldsymbol{E_\text{bi}} + \boldsymbol{E_\text{ext}}\). Таким образом, приводя к сдвигу и асимметрии ВАХ относительно нулевого напряжения (явление imprint). В случае наличия

% Встроенные поля приводят к расщеплению пиков переполяризационного тока на вольт-амперных характеристиках (ВАХ) за счёт

% Одной из причиной образования внутренних полей является наличие градиента механических напряжений (флексоэлектрический эффект)

Встроенное поле значительно влияет на функциональные свойства структур. Во-первых, при приложении внешнего электрического поля \(E_\text{ext}\), эффективное поле \(E_\text{eff}\) в СЭ слое складывается из суммы полей: \(\boldsymbol{E_\text{eff}} = \boldsymbol{E_\text{bi}} + \boldsymbol{E_\text{ext}}\), приводя к сдвигу и асимметрии ВАХ относительно нулевого напряжения (явление imprint). При наличии противоположно направленных встроенных полей в образце, происходит расщепление пиков переполяризационного тока на вольт-амперных характеристиках (ВАХ).

Во-вторых, при циклировании структуры происходит перераспределение кислородных вакансий и уменьшение встроенного поля, что приводит к депиннингу доменов и увеличению остаточной поляризации (wake-up эффект).

Таким образом, увеличение величины встроенного поля приводит к увеличению длительности wake-up эффекта, что \todo{приводит} к различиям в величине остаточной поляризации после одинакового циклирования структур с различной энергией импульса. Большее встроенное поле также приводит к измельчению доменов: эффект пиннинга доменов.