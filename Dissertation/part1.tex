\chapter{Сегнетоэлектрические материалы}\label{ch:ch1}
\section{Перовскитоподобные материалы/материалы перовскитной структуры}

\section{Сегнетоэлектрический оксид гафния}\label{sec:ch1/sec1}
При нормальных условиях в оксиде гафния энергетически выгодной является неполярная моноклинная фаза \(P2_1/c\), неполярная орторомбическая \(Pbca\), полярная орторомбическая \(Pca2_1\). Однако, при определённых условиях возможна стабилизация полярной (сегнетоэлектрической) орторомбической фазы \(Pca2_1\). Сегнетоэлектрические свойства при этом определяются отсутствием центральной симметрии в кристаллической решётке.
Было показано, что за снижение свободной энергии сегнетоэлектрической фазы и её стабилизацию в оксиде гафния отвечает ряд характеристик. Так, необходим существенный вклад поверхностной энергии в свободную энергию, который достигается при достаточном уменьшении толщины плёнки. Кроме того, значительным фактором, влияющим на стабилизацию фазы являются механические напряжения в оксиде гафния. Известно, что в отсутствии верхнего электрода сегнетоэлектрические свойства структур на основе оксида гафния значительно ухудшаются. Этот эффект объясняется тем, что при быстрой термической обработке структуры с нижним и верхним электродами, в оксиде гафния возникают механические напряжения, которые образуются за счёт различия в коэффициенте теплового расширения материалов электродов и оксида гафния и приводят к образованию орторомбической фазы за счёт подавления образования моноклинной фазы. Наконец, важнейшим элементом является легирование оксида. Сегнетоэлектрические свойства разной степени выраженности были получены в плёнках с примесями Si \cite{}, Zr \cite{}, La \cite{}, Gd \cite{}, Y \cite{}, Ga \cite{chouprikNanoscaleDopingIts2022}. Несмотря на широкий спектр возможностей по выбору примесного элемента, использование большинства из указанных элементов приводит к повышению температуры кристаллизации плёнки значительно выше 400 \si{\degreeCelsius}, тем самым нарушая возможную совместимость с backend-of-line (BEOL) процессом в CMOS технологии \cite{schmitzLowTemperatureThin2018}, существенно затрудняя внедрение устройств памяти на основе оксида гафния в производство. Одним из наиболее перспективных материалов для создания энергонезависимой памяти является твёрдый раствор Hf0.5Zr0.5O2. Обладая сравнительно невысокой температурой кристаллизации \(\approx\) \SI{400}{\degreeCelsius}, и одинаковой долей элементов Hf и Zr, позволяющей контролировать стехиометрию с большой точностью, HZO ... структуры на основе имеют ... остаточной поляризации (или не надо про это пока)
\section{Ссылки}\label{sec:ch1/sec2}
Сошлёмся на библиографию.
% Для бла-бла было произведено моделирование
Для моделирования взаимодействия частиц платины с HZO были произведены квантовомеханические расчёты столкновений с помощью пакета SRIM \todo{ref}. Оценка энергии налетаемых частиц была получена с помощью расчёта кинетической энергии атомов при лазерной абляции:
\todo{добавить интуиции к формулке}
\[\bar{E} = 9.92 \cdot 10^{4} A^{1/8} \tau^{1/4} Z^{3/4} (Z+1)^{3/8} (I\lambda)^{1/2}k,\] где \(A\) -- атомная масса распыляемого элемента, \(\tau\) -- длительность импульса, \(Z\) -- средний заряд ионов, вылетающих при абляции, \(I\) -- интенсивность лазерного излучения в \si{\watt}/\si{\cm}\(^2\), \(\lambda\) -- длина волны лазера, \(k\) -- постоянная Больцмана.
При импульсном лазерном напылении использовался лазер Nd:YAG с длиной волны \(\lambda=\)\SI{1064}{\nano\meter}, длительностью импульсов \(\tau=\) \SI{10}{\nano\second}. Интенсивность оценивалась как \(I=\alpha(1-\beta)\frac{J}{S}\), где \(\alpha=(1-R)^{2N}\) -- коэффициент прохождения излучения через \(N=5\) элементов оптической системы,
изготовленных из кварцевого стекла с отражательной способностью \(R=0.0337\) \cite{polyanskiyRefractiveindexInfoDatabase2024} для длины волны \(\lambda\) (поглощение при этом не учитывалось), \(\beta=0.971\) -- отражательная способность платины \cite{polyanskiyRefractiveindexInfoDatabase2024}, \(J\) -- энергия одного импульса, \(S\) -- площадь лазерного пятна.
% Одна ссылка: \cite[с.~54]{Sokolov}\cite[с.~36]{Gaidaenko}.
% Две ссылки: \cite{Sokolov,Gaidaenko}.
% Ссылка на собственные работы: \cite{vakbib1, confbib2}.
% Много ссылок: %\cite[с.~54]{Lermontov,Management,Borozda} % такой «фокус»
%вызывает biblatex warning относительно опции sortcites, потому что неясно, к
%какому источнику относится уточнение о страницах, а bibtex об этой проблеме
%даже не предупреждает
% \cite{Lermontov, Management, Borozda, Marketing, Constitution, FamilyCode,
%     Gost.7.0.53, Razumovski, Lagkueva, Pokrovski, Methodology, Berestova,
%     Kriger}%
% \ifnumequal{\value{bibliosel}}{0}{% Примеры для bibtex8
%     \cite{Sirotko, Lukina, Encyclopedia, Nasirova}%
% }{% Примеры для biblatex через движок biber
%     \cite{Sirotko2, Lukina2, Encyclopedia2, Nasirova2}%
% }%
% .
% И~ещё немного ссылок:~\cite{Article,Book,Booklet,Conference,Inbook,Incollection,Manual,Mastersthesis,
%     Misc,Phdthesis,Proceedings,Techreport,Unpublished}
% % Следует обратить внимание, что пробел после запятой внутри \cite{}
% % обрабатывается ожидаемо, а пробел перед запятой, может вызывать проблемы при
% % обработке ссылок.
% \cite{medvedev2006jelektronnye, CEAT:CEAT581, doi:10.1080/01932691.2010.513279, doi:10.1021/acsami.0c16741,
%     Gosele1999161,Li2007StressAnalysis, Shoji199895, test:eisner-sample,
%     test:eisner-sample-shorted, AB_patent_Pomerantz_1968, iofis_patent1960}%
% \ifnumequal{\value{bibliosel}}{0}{% Примеры для bibtex8
% }{% Примеры для biblatex через движок biber
%     \cite{patent2h, patent3h, patent2}%
% }%
% .


\nomenclature{FEM}{finite element method, метод конечных элементов}
\nomenclature{HZO}{Hf0.5Zr0.5O2, оксид гафния-циркония}
\nomenclature{АСО}{атомно-слоевое осаждение}
\nomenclature{АСМ}{атомно-силовая микроскопия}
\nomenclature{PUND}{positive up negative down, электрофизический метод измерения остаточной поляризации}
\nomenclature{CMOS}{complementary metal-oxide-semiconductor, комплементарная структура металл-оксид-полупроводник}
\nomenclature{БТО}{быстрая термическая обработка}
\nomenclature{NVM}{non-volatile memory, энергонезависимая память}
\nomenclature{FIT}{finite integration technique, метод конечных интегралов}
\nomenclature{FMM}{fast multipole method, быстрый метод многополюсника}
\nomenclature{FVTD}{finite volume time-domain, метод конечных объёмов
    во~временной области}
\nomenclature{MLFMA}{multilevel fast multipole algorithm, многоуровневый
    быстрый алгоритм многополюсника}
\nomenclature{BEM}{boundary element method, метод граничных элементов}
\nomenclature{CST MWS}{Computer Simulation Technology Microwave Studio
    программа для компьютерного моделирования уравнен Максвелла}
\nomenclature{DDA}{discrete dipole approximation, приближение дискретиных
    диполей}

\FloatBarrier
