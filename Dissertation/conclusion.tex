\chapter*{Заключение}                       % Заголовок
\addcontentsline{toc}{chapter}{Заключение}  % Добавляем его в оглавление

%% Согласно ГОСТ Р 7.0.11-2011:
%% 5.3.3 В заключении диссертации излагают итоги выполненного исследования, рекомендации, перспективы дальнейшей разработки темы.
%% 9.2.3 В заключении автореферата диссертации излагают итоги данного исследования, рекомендации и перспективы дальнейшей разработки темы.
%% Поэтому имеет смысл сделать эту часть общей и загрузить из одного файла в автореферат и в диссертацию:

Основные результаты работы заключаются в следующем.
%% Согласно ГОСТ Р 7.0.11-2011:
%% 5.3.3 В заключении диссертации излагают итоги выполненного исследования, рекомендации, перспективы дальнейшей разработки темы.
%% 9.2.3 В заключении автореферата диссертации излагают итоги данного исследования, рекомендации и перспективы дальнейшей разработки темы.
В работе было проведено исследование возможных физических механизмов, влияющих на сегнетоэлектрические свойства конденсаторов на основе оксида гафния-циркония. На примере исследования микроскопических и функциональных свойств структур W/Hf\(_{0.5}\)Zr\(_{0.5}\)O\(_2\)/Pt методами АСМ, РЭМ, XRD, PFM и электрофизическими измерениями было показано, что параметры, используемые при импульсном лазерном напылении могут существенно влиять на функциональные свойства ячеек сегнетоэлектрической памяти.

Полученные в работе экспериментальные результаты: увеличение размера доменов, величины встроенного электрического поля и остаточной поляризации были объяснены на примере двух моделей. Во-первых, была предложена гипотеза о влиянии размера зёрен Pt на сегнетоэлектрические свойства структур. Так, увеличение размера зёрен приводит к уменьшению длины границ зёрен, через которые возможно проникновение атомов кислорода, тем самым снижая образование кислородных вакансий. В свою очередь, меньшая концентрация заряженных дефектов, которыми являются вакансии, приводит к образованию меньших встроенных полей. Встроенные поле же, как известно, могут значительно изменять сегнетоэлектрические свойства структур, в частности приводя к различной выраженности эффектов wake-up и imprint.

В качестве альтернативной модели, различие в величине встроенных полей было объяснено на примере влияния флексоэлектрического эффекта: с помощью моделирования процесса имплантирования методом Монте-Карло показано, что импульсное лазерное напыление платины может вызывать возникновение градиента механических напряжений в сегнетоэлектрическом слое, приводя к возникновению встроенного поля. При этом показано, что градиент механических напряжений увеличивается при увеличении энергии импульса, используемого при напылении, за счёт уширения распределения атомов платины в плёнке оксида гафния-циркония.

Однако, в работе не было отражено явное превалирование одного из механизмов. Тем самым, можно подчеркнуть необходимость дальнейших исследований для выяснения истинной микроскопической природы функциональных свойств ячеек сегнетоэлектрической памяти на основе тонких плёнок оксида гафния-циркония. Тем не менее, полученные результаты могут быть использованы для улучшения сегнетоэлектрических свойств структур с помощью оптимизации технологического процесса.

В работе также был предложен метод для количественного сравнения вертикальной и латеральной компоненты пьезоотклика при измерениях с помощью микроскопии пьезоотклика. Полученная методика была проверена на эксперименте и может быть использована для восстановления трёхмерного вектора поляризации в различного рода структурах, что может быть полезным как в исследовании природы сегнетоэлектрических свойств в новых материалах, так и в более прикладных задачах.

И какая-нибудь заключающая фраза.

Автор выражает благодарность своему научному руководителю А.\,А.~Чуприк за помощь в выборе тематики исследований, трактовке полученных экспериментальных результатов, а также за всяческую поддержку в течение работы в лаборатории. Также автор благодарит И.\,А.~Савичева за помощь в погружении в исследуемую область в начале своего пути и за обсуждение различных вопросов, возникавших в течение научной деятельности. Автор также благодарит М.\,В.~Спиридонова за помощь в освоении микроскопии пьезоотклика и устранении различных технических неполадок, возникавших в ходе экспериментов.

Последний параграф может включать благодарности.  В заключение автор
выражает благодарность и большую признательность научному руководителю
Иванову~И.\,И. за поддержку, помощь, обсуждение результатов и~научное
руководство. Также автор благодарит Сидорова~А.\,А. и~Петрова~Б.\,Б.
за помощь в~работе с~образцами, Рабиновича~В.\,В. за предоставленные
образцы и~обсуждение результатов, Занудятину~Г.\,Г. и авторов шаблона
*Russian-Phd-LaTeX-Dissertation-Template* за~помощь в оформлении
диссертации. Автор также благодарит много разных людей
и~всех, кто сделал настоящую работу автора возможной.
